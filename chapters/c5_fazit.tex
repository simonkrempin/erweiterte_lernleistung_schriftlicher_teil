\chapter{Fazit}
Dieses Kapitel schließt die erweiterte Lernleistung ab, indem auf die erarbeitete Leistung zurückgeblickt und beurteilt wird. Dazu wird in Abschnitt \ref{section:vergleich} die zuvor geplante Applikation mit dem zum Abgabezeitpunkt stehenden Produkt, im folgenden auch als finale Version bezeichnet, anhand von Features und User Interface verglichen. Des Weiteren werden die nicht implementierten Features thematisiert und besprochen. Danach folgt in Abschnitt \ref{section:anwendungsbeispiel} ein Beispiel, wie die Applikation zum Zeitpunkt der Abgabe genutzt werden kann. Im dritten Abschnitt wird meine Arbeitsweise in Bezug auf die Applikation beschrieben und in Hinblick auf die Effektivität bewertet. Abschließend behandelt Abschnitt \ref{weiteresvorgehen}, wie das weitere Vorgehen aussieht. Das schließt die weitere Implementierung von Features, die Entwicklung von neuen Features und die Veröffentlichung der Applikation ein.

\section{Gegenüberstellung von Planung und finalen Produkt}
\label{section:vergleich}

Dadurch, dass die Applikation zeitlich bedingt nicht komplett fertig geworden ist, lasst sich als erstes festhalten, dass einige Features, welche als wichtig einzustufen sind, noch fehlen. Dennoch ist es möglich für Nutzer die Kernfunktion, das Lernen von Karten, zu nutzten.

Nun zum Vergleich der Benutzeroberfläche. Direkt vorweg zu erwähnen ist, dass die meisten Layouts und Objekte so aussehen, wie vorher designt. Als passendes Beispiel ist der Startbildschirm zu nennen, welcher in Design und fertiger Applikation identisch aussieht.

Die Unterschiede halten sich, wie schon erwähnt, minimal. Als erstes ist das erstellen von Karten und Ordnern zu nennen, denn die Ansicht wurde in der finalen Version um Infotexte ergänzt. Auch wenn diese Änderung nicht essentiell war, bot sie sich im Laufe der Entwicklung an und verbessert dazu noch das Verständnis der Nutzer. Komplett aus der Applikation wurde die Ansicht von Ordnern nur mit Karten gelöscht. Was während der Planung noch wie ein logisches Feature schien, stellte sich als sinnlos heraus und wirkte gegen eine verständliche und einheitliche Anwendung.

\section{Anwendungsbeispiel}
\label{section:anwendungsbeispiel}

Die Applikation lässt sich in der finalen Version divers nutzen. Im folgenden wird ein Beispiel durchlaufen, wie es aussehen könnte, wenn ein Nutzer wichtige Stellen aus einem Buch in die Anwendung übertragen möchte.

Sobald die Anwendung gestartet wurde, wird dem Nutzer die Startseite präsentiert. Auf dieser kann ein Ordner für das Buch erstellt werden, in welchen die wichtigen Stellen gespeichert werden. Bis jetzt hat der Nutzer drei Stellen aus dem Buch markiert, die auf unterschiedliche Weise abgespeichert werden müssen. Im Fall von einem unbekannten Wort, wird das Wort auf der Vorderseite und die Erklärung auf die Rückseite geschrieben. Außerdem hat der Nutzer einen Satz markiert, welchen er sich merken möchte. Da der Inhalt nur aus einer Komponente besteht, wird nur die Vorderseite benutzt. Der Lerneffekt bleibt der selbe, da der Inhalt trotzdem wiederholt werden kann. Zu guter letzt speichert der Nutzer ein Zitat auf der Vorderseite der Karte und den Verfasser auf der Rückseite. Nun können die Karten in Abständen wiederholt werden, damit der Inhalt gefestigt wird.

\section{Eigene Beurteilung der Arbeitsweise}
Während der Planung und Implementierung habe ich mit zahlreichen Hilfstools gearbeitet. Das \textit{User Interface} ist mithilfe von \textit{Adobe XD} entstanden. \textit{Adobe XD} ist ein in der Branche viel genutztes Werkzeug um \textit{User Interfaces} zu designen und arbeitet mit Boxen und Kreisen, welche sich auf eine Arbeitsfläche ziehen und graphisch verändern lassen. Die Datenbank wurde in dem Online Tool \textit{dbdiagram.io} erstellt, womit durch simple Code-Anweisungen \textit{Relationale-Datenbank-Schemata} erstellt werden können. \textit{dbdiagram.io} biete eine übersichtliche Ansicht von Tabellen und wie diese mit einander arbeiten, indem Primärschlüssel hervorgehoben, Fremdschlüssel mit ihren referierten Attribut verbunden, Datentypen dargestellt und Kardinalitäten angezeigt werden. In \textit{Milanote} wurden alle Schritte geplant und aufgeschrieben. Die Einzelnen Schritte sind in Versionen aufgeteilt, um den Überblick zu behalten. Des Weiteren konnten Beziehungen von verschiedenen Ansichten durch \textit{Milanote} anschaulich dargestellt werden. Zu guter letzt wurde der Code der Applikation mit Github verwaltet. Dabei wurden Features jedes mal als eigene Commits gespeichert um einen optimale Kontrolle über die Versionen zu haben.

Rückblickend gelang es mir nicht immer Github wie geplant zu nutzten, denn implementierte Features sammelten sich öfter Mal ohne als Version abgespeichert zu werden. Dazu habe ich mich öfters nicht an den vorher erstellten Plan gehalten, sondern habe Features fürher oder später implementiert, als angesetzt. Das hatte zur Folge, dass die To-Do Listen und Versionen, welche in der Übersichtlichkeit helfen sollten, chaotisch und unübersichtlich waren. Zum Schluss lief das jedoch deutlich besser und bringt Hoffnung, dass es bei meinem nächsten Projekt ohne Probleme funktioniert. Die Auswahl der Tools für die Entwicklung ist wiederum sehr gut gewählt und ich werde sie mir für weitere Projekte merken.

\section{Weiteres Vorgehen} \label{weiteresvorgehen}
Als erstes soll die Applikation in ihrer \textit{minimal vital product} (\textit{MVP}) Version auf Android und IOS veröffentlichen werden. Welche Features die \textit{MVP} Version umfasst, werde ich nach der Abgabe dieser Ausarbeitung definieren. Mit dem \textit{MVP} werde ich dann versuchen möglichst viele Anwender zu erreichen, um Nutzer zu gewinnen und Feedback zu erhalten.

Nachdem das erledigt ist, fokussiere ich mich auf das Grundgerüst der Anwendung, welches noch viel Potential für weitere Features bietet, wie die zwei Ideen, welche sich im Laufe der Entwicklung ergeben haben. Als erstes soll das Lernen Modul ausgebaut werden. Durch eine große Datenlage sollen auch Abfragen ermöglicht werden, indem Lösungen von anderen Karten als mögliche Antworten mit der richten Antwort angezeigt werden. Der Nutzer muss in diesem Fall die richtige Lösung auswählen. Ebenfalls lässt sich über Abfragen nachdenken, in welchen der Nutzer die Lösungen eintippen muss. Eine weitere Möglichkeit wäre eine Abfrage von Zeichnungen, in welchem mithilfe von dem Finger Zeichen oder Sonstige Symbole nach gemalt werden. Das kann besonders Hilfreich sein, wenn neue Sprachen mit einem anderen Buchstabensystem gelernt werden. Die letzte Möglichkeit, welche im mir im Laufe der Entwicklung eingefallen ist, wären Lückentexte. An den aufgelisteten Abfragemöglichkeiten, wird deutlich, dass das Potenzial noch nicht vollkommen ausgenutzt wurde und das Abfragemodul um Aspekte erweitert werden kann.

Das gilt auch für das Angebot an Lernmöglichkeiten. Selbstgeführte Kurse lassen sich mithilfe der Online Ordner zwar verwirklichen, aber sind noch nicht optimal. Die Anwendung lässt sich dahingegen noch erweitern, dass auch Nutzer, welche ihr Wissen ausbauen möchten, aber zeitlich gerne auf vorgefertigte Lernmaterialen zurückgreifen. Diese Idee lässt sich noch erweitern, indem die Kurse gekauft werden können und so Nutzer, welche ihr Lernmaterial veröffentlichen, einen kleinen Nebenverdienst erwirtschaften können. Im Sinne der Steigerung des eigenen Profits, kann das Online Tool möglichst billig angeboten werden, um den Kosten des Server entgegenzuwirken. Im Zuge einer Monetarisierung, müssten die Rechtlichen Bedingungen jedoch geändert werden. Eine Software für die Eltern, um Kindern Inhalte freizuschalten und Einkäufe zu blockieren, wäre inzuge, dass jedes Alter von der Applikation abgedeckt werden soll, von Nöten.
