\chapter{Fazit}
\section{Gegenüberstellung von Planung und finalen Produkt}
\section{Anwendungsbeispiel}
\section{Eigene Beurteilung der Arbeitsweise}
Während der Planung und Implementierung habe ich mit zahlreichen Hilfstools gearbeitet. Das \textit{User Interface} ist mithilfe von \textit{Adobe XD} entstanden. \textit{Adobe XD} ist ein in der Branche viel genutztes Werkzeug um \textit{User Interfaces} zu designen und arbeitet mit Boxen und Kreisen, welche sich auf eine Arbeitsfläche ziehen und graphisch verändern lassen. Die Datenbank wurde in dem Online Tool \textit{dbdiagram.io} erstellt, womit durch simple Code-Anweisungen \textit{Relationale-Datenbank-Schemata} erstellt werden können. \textit{dbdiagram.io} biete eine übersichtliche Ansicht von Tabellen und wie diese mit einander arbeiten, indem Primärschlüssel hervorgehoben, Fremdschlüssel mit ihren referierten Attribut verbunden, Datentypen dargestellt und Kardinalitäten angezeigt werden. In \textit{Milanote} wurden alle Schritte geplant und aufgeschrieben. Die Einzelnen Schritte sind in Versionen aufgeteilt, um den Überblick zu behalten. Des Weiteren konnten Beziehungen von verschiedenen Ansichten durch \textit{Milanote} anschaulich dargestellt werden. Zu guter letzt wurde der Code der Applikation mit Github verwaltet. Dabei wurden Features jedes mal als eigene Commits gespeichert um einen optimale Kontrolle über die Versionen zu haben.

Rückblickend gelang es mir nicht immer Github wie geplant zu nutzten. Implementierte Features sammelten sich öfter Mal ohne als Version abgespeichert zu werden. Dazu habe ich mich öfters nicht an den Plan gehalten. Features wurden in einem Zug mit andern implementiert, obwohl diese erst später folgen sollten. Dadurch waren Versionen und To-Do Listen zwischen durch chaotisch. Zum Schluss lief das jedoch deutlich besser und bringt Hoffnung, dass es bei meinem nächsten Projekt ohne Probleme funktioniert. Die Auswahl der Tools für die Entwicklung ist wiederum sehr gut gewählt und ich werde sie mir für weitere Projekte merken.

\section{Weiteres Vorgehen}
Als erstes soll die Applikation in ihrer \textit{minimal vital product} (\textit{MVP}) Version auf Android und IOS veröffentlichen werden. Welche Features die \textit{MVP} Version umfasst, werde ich nach der Abgabe dieser Ausarbeitung definieren. Mit dem \textit{MVP} werde ich dann versuchen möglichst viele Anwender zu erreichen, um Nutzer zu gewinnen und Feedback zu erhalten. Das verwirkliche ich, indem ich Schulen die moderne Lernmethode vermittle.

Das Grundgerüst der Anwendung bietet noch viel Potenzial für weitere Features. So auch die zwei Ideen, welche sich im Laufe der Entwicklung ergeben haben. Als erstes soll das Lernen Modul ausgebaut werden. Durch eine große Datenlage sollen auch Abfragen ermöglicht werden, indem Lösungen von anderen Karten als mögliche Antworten mit der richten Antwort angezeigt werden. Der Nutzer muss in diesem Fall die richtige Lösung auswählen. Ebenfalls lässt sich über Abfragen nachdenken, in welchen der Nutzer die Lösungen eintippen muss. Eine weitere Möglichkeit wäre eine Abfrage von Zeichnungen, in welchem mithilfe von dem Finger Zeichen oder Sonstige Symbole nach gemalt werden. Das kann besonders Hilfreich sein, wenn neue Sprachen mit einem anderen Buchstabensystem gelernt werden. Die letzte Möglichkeit, welche im mir im Laufe der Entwicklung eingefallen ist, wären Lückentexte. An den aufgelisteten Abfragemöglichkeiten, wird deutlich, dass das Potenzial noch nicht vollkommen ausgenutzt wurde und das Abfragemodul um Aspekte erweitert werden kann.

Das gilt auch für das Angebot an Lernmöglichkeiten. Selbstgeführte Kurse lassen sich mithilfe der Online Ordner zwar verwirklichen, aber sind noch nicht optimal. Die Anwendung lässt sich dahingegen noch erweitern, dass auch Nutzer, welche ihr Wissen ausbauen möchten, aber zeitlich gerne auf vorgefertigte Lernmaterialen zurückgreifen. Diese Idee lässt sich noch erweitern, indem die Kurse gekauft werden können und so Nutzer, welche ihr Lernmaterial veröffentlichen, einen kleinen Nebenverdienst erwirtschaften können. Im Sinne der Steigerung des eigenen Profits, kann das Online Tool möglichst billig angeboten werden, um den Kosten des Server entgegenzuwirken. Im Zuge einer Monetarisierung, müssten die Rechtlichen Bedingungen jedoch geändert werden. Eine Software für die Eltern, um Kindern Inhalte freizuschalten und Einkäufe zu blockieren, wäre inzuge, dass jedes Alter von der Applikation abgedeckt werden soll, von Nöten.
