\pagenumbering{arabic}
\setcounter{page}{3}
\chapter{Einleitung}
In der Schule wird täglich gelernt, dennoch wird oftmals nicht gelernt, wie man lernt. Ein Schüler muss selber herausfinden, wie er relevante Informationen möglichst zeiteffizient und detailliert beibehält. Aber gerade die richtigen Lernmethoden machen oft den Unterschied zwischen guten und schlechten Schülern aus \cite{OECD:Lernen}\cite{Hattie:Einflussfaktoren}. Es ist daher essenziell diese Unterschiede auszugleichen und damit einen Beitrag für eine größere Chancengleichheit zu leisten. Im besten Fall sorgt eine Verbesserung der Lernmethoden langfristig zu einem, wenn auch nur minimalen, höheren Wissensniveau.

Es ist nicht erst seit der von Ebbinghaus formulierten Forgetting Curve bekannt, dass mangelnde Wiederholung unmittelbar zu einem Verlust von Wissen führen kann. Das ist besonders kritisch, wenn andere Gedanken und Themen darauf aufbauen. Deswegen ist es essenziell, die Grundsteine zu festigen. Methoden, wie man das ermöglicht, sind unter Schülern jedoch nicht weit verbreitet. Ein Großteil von Schülern nutzt \textit{Cramming} als Methode, um sich das für eine Klausur notwendige Wissen kurzfristig anzueignen. Aufgrund des kurzen Zeitraums und fehlender Wiederholung landet dieses Wissen jedoch zu großen Teilen ausschließlich im Kurzzeitgedächtnis, so dass dieses nach der Klausur dem Schüler nicht länger zur Verfügung steht. Diese Methode ist für das tiefe Verstehen und Verknüpfen von Themen suboptimal. Folglich ist eine neue Art zu lernen von Nöten, um den Lernprozess zu optimieren und Schüler zu entlasten.

Die Idee, Schülern eine verbesserte Lernmethode nahe zu bringen, ist nicht neu. Andere Applikationen zur Unterstützung des Lernprozess sind schon im Umlauf. Die bekannteste ist AnkiDroid. Diese vermitteln zwar weitestgehend richtig die Methoden, aber es liegt, im Gegensatz zu der in dieser Arbeit entwickelten Applikation, kein Fokus auf der Schule und den Unterricht. Aus diesem Grund ist der Mehrwert für Schüler mit jetzigen Mitteln noch nicht optimal ausgenutzt, weshalb mithilfe der Ergebnisse der besonderen Lernleistung das Potenzial für Schüler maximiert werden soll.

In dieser Ausarbeitung wird gezeigt, dass durch Wiederholung in immer länger werdenden Intervallen bessere Ergebnisse erzielt werden können. Aufgrund dieser Erkenntnis wird eine Software geschrieben, welchen den Lernprozess von Schülern unterstützt, indem sie \textit{spaced repetition} anwendet, um den Memorierstoff nahezubringen. Ergänzt wird dies durch Erörterung gewählter Ansätze und begegneter Hürden, welche sich während der Entwicklung und Recherche ergeben haben. \par
