\chapter{Implementierung}
\label{chapter:programmierung}
In diesem Kapitel werden die Implementierungsdetails der Applikation diskutiert. Das fundamentale Prinzip der Benutzeroberfläche wird in Abschnitt \ref{section:benutzeroberfläche} erörtert. Danach folgt in Abschnitt \ref{section:interne_logik} die Umsetzung der Datenbank und Nutzungsweise der Applikation. Des Weiteren wird über die Implementierung des Servers in Abschnitt \ref{section:datenaustausch} gesprochen. Schlussendlich wird der Datenaustausch zwischen der Applikation und dem Server in Abschnitt \ref{section:datenaustausch} thematisiert, wo auf Logik des Server und der Anwendung eingegangen wird.

\section{Benutzeroberfläche}
\label{section:benutzeroberfläche}
Mit dem Prinzip der objektorientierten Programmierung werden Widgets in eigenen Dateien implementiert. Die dadurch entstehenden Vorteile sind die gleichen wie bei Methoden, denn gerade Elemente der Benutzeroberfläche haben viele repetitive Stellen und die wenigen Unterschiede können durch Parameter abgeändert werden. Damit lassen sich Änderungen leichter anwenden und der Code ist besser lesbar. Die einzige Hürde ist, die Widgets für so viele Anwendungsfälle wie möglich verwendbar zu machen.

Das Prinzip wird an der \texttt{BasicButton}-Klasse veranschaulicht, welche einen grünen Button mit einem weißen Text erstellt. Implementiert wird dieser Button beispielsweise in der Klasse \texttt{FolderSettingsView}, in welcher die Einstellung der Ordner geändert werden können. Nachdem die UI-Elemente für die Einstellungen erstellt werden, folgt am Ende der Ansicht der Button zum Speichern der Einstellungen. In den Button werden über die drei Parameter die nötigen Werte übertragen, von welchen zwei verpflichtend sind und lediglich einer optional ist. Es ist ein Parameter optional, weil der Text und Funktion des Buttons jedes Mal unterschiedlich sind und deswegen nicht optional sein kann. Der optionale Parameter mit einem Standardwert erweist sich als sinnvoll, da er bestimmt, ob der Button gedrückt werden kann, was in den meisten Szenarien der Fall ist. Mit den Parametern modifiziert, erstellt die Klasse \texttt{BasicButton} den Button für den Anwendungsfall. Beim Drücken des Buttons wird so die vorher mitgegebene Funktion ausgeführt und auch der Text des Buttons ist gleich mit dem vom Parameter definierten Wert.

Ein weiterer Vorteil wird beim Aktualisieren von Informationen ersichtlich, denn dabei werden nicht nur die Bereiche mit den Informationen aktualisiert, sondern das ganze Widget. Ohne Auskapselung der UI-Elemente muss bei einer Struktur wie in Abbildung \ref{fig:flutter_structure} das komplette \textit{root}-Widget neu gerendert werden und Rechenleistung für Widgets ohne Änderungen aufgewandt werden. Durch das Auskapseln werden die Widgets in ihren eigenen \texttt{build}-Methoden implementiert und können explizit aufgerufen werden, ohne jedes Element der Benutzeroberfläche neu zu rendern.

\begin{center}
    \begin{figure}
        \centering
        \includegraphics[width=0.5\textwidth]{images/layerstruktur_flutter.png}
        \caption{Überblick über die Flutter \textit{Widget}-Struktur}
        \label{fig:flutter_structure}
    \end{figure}
\end{center}

\section{Interne Logik}
\label{section:interne_logik}
Die Klasse \texttt{AppDatabase} implementiert die Verbindung zur Datenbank der Applikation. Dafür wird eine Referenz auf die Verbindung zur lokalen Datenbank gespeichert und bei Bedarf von der Methode \texttt{get} zurückgegeben. Sollte noch keine Verbindung zur Datenbank bestehen, da diese zuvor noch nicht benötigt worden ist, dann wird diese durch die Methode \texttt{\_initDatabase} vor dem Zurückgeben der Verbindung initialisiert und für spätere Anfragen gespeichert.

In der Methode \texttt{\_initDatabase}, wird in dem Speicherpfad überprüft, ob eine Datenbank vorliegt. Ist dies der Fall, führt die Methode \texttt{openDatabase} aus und knüpft damit eine Verbindung, ohne auf \texttt{onCreate} zurückzugreifen. Falls keine Datenbank vorliegt, wird mit der Methode \texttt{\_createDatabase} im \texttt{onCreate} Parameter eine neue Datenbank erstellt.

\texttt{\_createDatabase} führt alle SQL-Anfragen zum Erstellen des in Abbildung \ref{fig:appdatabase} dargestellten Datenbankschemas aus. Listing \ref{lst:_createDatabase} zeigt beispielhaft die für die Erstellung der card Tabelle notwendige SQL-Abfrage, die durch besondere Schlüsselwörter ergänzt wurde. Unteranderem wurde \texttt{CREATE TABLE} um \texttt{IF NOT EXISTS} ergänzt, um Fehler zu verhindern, die auftreten würden, wenn die Tabelle erstellt wird, obwohl es schon eine Tabelle \texttt{card} gibt. Die gespeicherten Karten lassen sich an ihrer eindeutigen \texttt{\_id} erkennen. IN SQL werden solche Werte als \texttt{PRIMARY KEY} gekennzeichnet, um das vorkommen von doppelten Werten zu verhindern. Damit der Primärschlüssel sich abhebt, wird vor den Wert ein Unterstricht gesetzt. Da \texttt{\_id} keinen Nutzen hat, außer die Karten auseinanderhalten zu können, wird mit dem Schlüsselwort \texttt{AUTOINCREMENT} ein Wert zugeweisen, welcher mit jedem Eintrag erhöht wird, womit verhindert wird, dass doppelte Werte auftreten.

Mit \texttt{execute} können SQL-Anfragen, im \texttt{String}-Formt, ausgeführt werden. Um die Lesbarkeit zu erhöhen, werden die Anfragen vorher in einem \texttt{String} definiert. Damit ist deutlich, was die einzelnen SQL-Anfragen machen und vereinfacht das Debugging und Erweitern der Anfragen.

\begin{lstlisting}[caption={Erstellen der Datenbank},language=sql,captionpos=b,label={lst:_createDatabase},numbers=left,frame=none,escapechar=|]
CREATE TABLE IF NOT EXISTS card(
    _id INTEGER PRIMARY KEY AUTOINCREMENT,
    root INTEGER,
    front TEXT,
    back TEXT,
    status INTEGER,
    public BOOLEAN
\end{lstlisting}

Zur Abstraktion bietet die \texttt{AppDatabase} darüber hinaus verschiedene Wrapper-Methoden an, welche spezifische SQL-Abfragen kapseln und somit zu einer erhöhten Lesbarkeit außerhalb der \texttt{AppDatabase}-Klasse beitragen. Darüber hinaus ergibt sich der Vorteil, dass alle SQL-Abfragen in der \texttt{AppDatabase}-Klasse gebündelt werden können. Ein Ausschnitt einer solchen Methode wird in Listing \ref{lst:methode_karten_wiedergeben} dargestellt. Beim Aufruf der Methode wird zunächst die Instanz der Datenbank nach dem vorherigen Prinzip erhalten. Auf die Datenbank wird eine Abfrage, welche alle Karten nach einer spezifischen Reihenfolge oder nach dem Erstellungsdatum eines Ordners sortiert, ausgeführt und das Ergebnis wird gespeichert. Schlussendlich wird das Ergebnis von einer Tabelle in eine Liste von Karten (\texttt{Card}) formatiert und zurückgegeben. Anzumerken ist, dass durch das Fragezeichen in Zeile \ref{zeile:fragezeichen}, wofür die Werte von \texttt{whereArgs:} eingefügt werden, \textit{SQL-Injections} verhindert werden. Ohne das verhindern von \textit{SQL-Injections} ist eine Datenbank leicht angreifbar, da zusätzliche SQL-Befehle in die eigentliche Anfrage injiziert werden können und so die Logik verändert wird. Dafür muss in ein Eingabefeld der Anwendung SQL Syntax geschrieben werden. Beim durchlaufen, wird die vom Nutzer gestellte SQL-Anfrage als eigene, valide Anfrage gesehen und ausgeführt. Die damit erreichbaren Schäden gehen von Daten löschen bis zum Einsehen von allen Daten auf der Datenbank. Durch \textit{Escaping} werden \textit{SQL-Injections} verhindert und da jede Anwendung vor dem gleichen Problem steht, vereinfacht Dart mit \texttt{whereArgs} die Arbeit, denn das herkömmliche aneinanderreihen von Zeichenfolgen mit \texttt{concat} verhindert das nicht.

\begin{lstlisting}[caption={Methode, welche alle Karten eines Ordners wiedergibt},captionpos=b,label={lst:methode_karten_wiedergeben},numbers=left,frame=none,escapechar=|]
Future<List<Card>> readCards(int folderId, String? sortingOrder) async {
  final db = await instance.database;

  final res = await db.query(
    'card',
    where: 'root = ?', |\label{zeile:fragezeichen}|
    whereArgs: [folderId],
    orderBy: sortingOrder ?? '_id ASC',
  );
  return res.isNotEmpty ? res.map((e) => Card.fromMap(e)).toList() : [];
}
\end{lstlisting}

\begin{lstlisting}[caption={Anfrage aus Listing \ref{lst:methode_karten_wiedergeben} als herkömmliche SQL-Anfrage},language=sql,captionpos=b,label={lst:sql_anfrage},numbers=left,frame=none,escapechar=|]
SELECT *
FROM card
WHERE root = folderId
ORDER BY CASE 
    WHEN sortingOrder is NULL THEN _id ASC 
    ELSE sortingOrder
END
\end{lstlisting}

Sofern möglich sollte auf die vorgefertigte Methode zurückgegriffen werden, da diese Performance optimiert und vor \textit{SQL-Injections} geschützt ist. Dazu kommt noch, dass die \texttt{query}-Methode in einer Vielzahl an Anwendungsfällen leichter zu lesen ist, besonders für Programmierer ohne SQL-Kenntnisse, wie an dem Beispiel \ref{lst:methode_karten_wiedergeben} erkennbar.

Die Methode in Listing \ref{lst:sql_anfrage} kann innerhalb des Programms aufgerufen werden. Dazu muss \texttt{AppDatabase} importiert werden und darauf kann die static Variable \texttt{instance} aufgerufen werden. Da \texttt{instance} eine Referenz auf die Klasse \texttt{AppDatabase} hat, kann der Aufruf um einen Methodenaufruf erweitertet werden. Die erhaltenen Daten können dann im UI angezeigt werden. Da das Erhalten der Daten jedoch dauert, muss im UI zwischen zwei Zuständen unterschieden werden. Im ersten Zustand wird ein Ladebalken angezeigt, welcher dem Nutzer vermittelt, dass die Anwendung im Hintergrund Daten überträgt und verarbeitet. Sobald die Daten geladen wurden, wird \texttt{\_contentLoading} auf \texttt{false} gesetzt und die Benutzeroberfläche aktualisiert. 

In Listing \ref{lst:deleting} wird eine Methode dargestellt, welche einen Ordner mit samt seiner Unterordner zu löschen. Dafür wird eine Liste an zu löschenden Ordnern erstellt, die zunächst nur den vom Benutzer ausgewählten Ordner enthält. Für jeden Ordner in der Liste werden alle Unterordner von der Datenbank gelesen und in die Liste eingefügt. Danach wird der Ordner und alle dazugehörigen Karten gelöscht.

\begin{algorithm}
	\caption{Algorithmus zum löschen von Ordnern samt Unterorder} 
	\begin{algorithmic}[1]
	    \State $db \leftarrow database$ $instance$
	    \State $elementsToDelete \leftarrow folderID$
		\While{$elementsToDelete\neq\emptyset$}
		    \State $currentId\leftarrow element \in$ $elementsToDelete$ and remove it
		    \State $subfolders\leftarrow$ all folders with $id=currentId$ from $db$
		    \For{$subfolder$ in $subfolders$}
		        \State add $subfolder$ to $elementsToDelete$
		    \EndFor
		    \State delete folder where $id=currentId$ from $db$
		    \State delete cards where $root=currentId$ from $db$
		\EndWhile
	\end{algorithmic} 
\end{algorithm}

\section{Backend}
\label{section:backend}
Hier so Backend :)

\section{Datenaustausch}
\label{section:datenaustausch}
Hier ist Datenaustausch am passieren glaube ich :)
